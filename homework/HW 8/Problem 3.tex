
    




    
\documentclass[11pt]{article}

    
    \usepackage[breakable]{tcolorbox}
    \tcbset{nobeforeafter} % prevents tcolorboxes being placing in paragraphs
    \usepackage{float}
    \floatplacement{figure}{H} % forces figures to be placed at the correct location
    
    \usepackage[T1]{fontenc}
    % Nicer default font (+ math font) than Computer Modern for most use cases
    \usepackage{mathpazo}

    % Basic figure setup, for now with no caption control since it's done
    % automatically by Pandoc (which extracts ![](path) syntax from Markdown).
    \usepackage{graphicx}
    % We will generate all images so they have a width \maxwidth. This means
    % that they will get their normal width if they fit onto the page, but
    % are scaled down if they would overflow the margins.
    \makeatletter
    \def\maxwidth{\ifdim\Gin@nat@width>\linewidth\linewidth
    \else\Gin@nat@width\fi}
    \makeatother
    \let\Oldincludegraphics\includegraphics
    % Set max figure width to be 80% of text width, for now hardcoded.
    \renewcommand{\includegraphics}[1]{\Oldincludegraphics[width=.8\maxwidth]{#1}}
    % Ensure that by default, figures have no caption (until we provide a
    % proper Figure object with a Caption API and a way to capture that
    % in the conversion process - todo).
    \usepackage{caption}
    \DeclareCaptionLabelFormat{nolabel}{}
    \captionsetup{labelformat=nolabel}

    \usepackage{adjustbox} % Used to constrain images to a maximum size 
    \usepackage{xcolor} % Allow colors to be defined
    \usepackage{enumerate} % Needed for markdown enumerations to work
    \usepackage{geometry} % Used to adjust the document margins
    \usepackage{amsmath} % Equations
    \usepackage{amssymb} % Equations
    \usepackage{textcomp} % defines textquotesingle
    % Hack from http://tex.stackexchange.com/a/47451/13684:
    \AtBeginDocument{%
        \def\PYZsq{\textquotesingle}% Upright quotes in Pygmentized code
    }
    \usepackage{upquote} % Upright quotes for verbatim code
    \usepackage{eurosym} % defines \euro
    \usepackage[mathletters]{ucs} % Extended unicode (utf-8) support
    \usepackage[utf8x]{inputenc} % Allow utf-8 characters in the tex document
    \usepackage{fancyvrb} % verbatim replacement that allows latex
    \usepackage{grffile} % extends the file name processing of package graphics 
                         % to support a larger range 
    % The hyperref package gives us a pdf with properly built
    % internal navigation ('pdf bookmarks' for the table of contents,
    % internal cross-reference links, web links for URLs, etc.)
    \usepackage{hyperref}
    \usepackage{longtable} % longtable support required by pandoc >1.10
    \usepackage{booktabs}  % table support for pandoc > 1.12.2
    \usepackage[inline]{enumitem} % IRkernel/repr support (it uses the enumerate* environment)
    \usepackage[normalem]{ulem} % ulem is needed to support strikethroughs (\sout)
                                % normalem makes italics be italics, not underlines
    \usepackage{mathrsfs}
    

    
    % Colors for the hyperref package
    \definecolor{urlcolor}{rgb}{0,.145,.698}
    \definecolor{linkcolor}{rgb}{.71,0.21,0.01}
    \definecolor{citecolor}{rgb}{.12,.54,.11}

    % ANSI colors
    \definecolor{ansi-black}{HTML}{3E424D}
    \definecolor{ansi-black-intense}{HTML}{282C36}
    \definecolor{ansi-red}{HTML}{E75C58}
    \definecolor{ansi-red-intense}{HTML}{B22B31}
    \definecolor{ansi-green}{HTML}{00A250}
    \definecolor{ansi-green-intense}{HTML}{007427}
    \definecolor{ansi-yellow}{HTML}{DDB62B}
    \definecolor{ansi-yellow-intense}{HTML}{B27D12}
    \definecolor{ansi-blue}{HTML}{208FFB}
    \definecolor{ansi-blue-intense}{HTML}{0065CA}
    \definecolor{ansi-magenta}{HTML}{D160C4}
    \definecolor{ansi-magenta-intense}{HTML}{A03196}
    \definecolor{ansi-cyan}{HTML}{60C6C8}
    \definecolor{ansi-cyan-intense}{HTML}{258F8F}
    \definecolor{ansi-white}{HTML}{C5C1B4}
    \definecolor{ansi-white-intense}{HTML}{A1A6B2}
    \definecolor{ansi-default-inverse-fg}{HTML}{FFFFFF}
    \definecolor{ansi-default-inverse-bg}{HTML}{000000}

    % commands and environments needed by pandoc snippets
    % extracted from the output of `pandoc -s`
    \providecommand{\tightlist}{%
      \setlength{\itemsep}{0pt}\setlength{\parskip}{0pt}}
    \DefineVerbatimEnvironment{Highlighting}{Verbatim}{commandchars=\\\{\}}
    % Add ',fontsize=\small' for more characters per line
    \newenvironment{Shaded}{}{}
    \newcommand{\KeywordTok}[1]{\textcolor[rgb]{0.00,0.44,0.13}{\textbf{{#1}}}}
    \newcommand{\DataTypeTok}[1]{\textcolor[rgb]{0.56,0.13,0.00}{{#1}}}
    \newcommand{\DecValTok}[1]{\textcolor[rgb]{0.25,0.63,0.44}{{#1}}}
    \newcommand{\BaseNTok}[1]{\textcolor[rgb]{0.25,0.63,0.44}{{#1}}}
    \newcommand{\FloatTok}[1]{\textcolor[rgb]{0.25,0.63,0.44}{{#1}}}
    \newcommand{\CharTok}[1]{\textcolor[rgb]{0.25,0.44,0.63}{{#1}}}
    \newcommand{\StringTok}[1]{\textcolor[rgb]{0.25,0.44,0.63}{{#1}}}
    \newcommand{\CommentTok}[1]{\textcolor[rgb]{0.38,0.63,0.69}{\textit{{#1}}}}
    \newcommand{\OtherTok}[1]{\textcolor[rgb]{0.00,0.44,0.13}{{#1}}}
    \newcommand{\AlertTok}[1]{\textcolor[rgb]{1.00,0.00,0.00}{\textbf{{#1}}}}
    \newcommand{\FunctionTok}[1]{\textcolor[rgb]{0.02,0.16,0.49}{{#1}}}
    \newcommand{\RegionMarkerTok}[1]{{#1}}
    \newcommand{\ErrorTok}[1]{\textcolor[rgb]{1.00,0.00,0.00}{\textbf{{#1}}}}
    \newcommand{\NormalTok}[1]{{#1}}
    
    % Additional commands for more recent versions of Pandoc
    \newcommand{\ConstantTok}[1]{\textcolor[rgb]{0.53,0.00,0.00}{{#1}}}
    \newcommand{\SpecialCharTok}[1]{\textcolor[rgb]{0.25,0.44,0.63}{{#1}}}
    \newcommand{\VerbatimStringTok}[1]{\textcolor[rgb]{0.25,0.44,0.63}{{#1}}}
    \newcommand{\SpecialStringTok}[1]{\textcolor[rgb]{0.73,0.40,0.53}{{#1}}}
    \newcommand{\ImportTok}[1]{{#1}}
    \newcommand{\DocumentationTok}[1]{\textcolor[rgb]{0.73,0.13,0.13}{\textit{{#1}}}}
    \newcommand{\AnnotationTok}[1]{\textcolor[rgb]{0.38,0.63,0.69}{\textbf{\textit{{#1}}}}}
    \newcommand{\CommentVarTok}[1]{\textcolor[rgb]{0.38,0.63,0.69}{\textbf{\textit{{#1}}}}}
    \newcommand{\VariableTok}[1]{\textcolor[rgb]{0.10,0.09,0.49}{{#1}}}
    \newcommand{\ControlFlowTok}[1]{\textcolor[rgb]{0.00,0.44,0.13}{\textbf{{#1}}}}
    \newcommand{\OperatorTok}[1]{\textcolor[rgb]{0.40,0.40,0.40}{{#1}}}
    \newcommand{\BuiltInTok}[1]{{#1}}
    \newcommand{\ExtensionTok}[1]{{#1}}
    \newcommand{\PreprocessorTok}[1]{\textcolor[rgb]{0.74,0.48,0.00}{{#1}}}
    \newcommand{\AttributeTok}[1]{\textcolor[rgb]{0.49,0.56,0.16}{{#1}}}
    \newcommand{\InformationTok}[1]{\textcolor[rgb]{0.38,0.63,0.69}{\textbf{\textit{{#1}}}}}
    \newcommand{\WarningTok}[1]{\textcolor[rgb]{0.38,0.63,0.69}{\textbf{\textit{{#1}}}}}
    
    
    % Define a nice break command that doesn't care if a line doesn't already
    % exist.
    \def\br{\hspace*{\fill} \\* }
    % Math Jax compatibility definitions
    \def\gt{>}
    \def\lt{<}
    \let\Oldtex\TeX
    \let\Oldlatex\LaTeX
    \renewcommand{\TeX}{\textrm{\Oldtex}}
    \renewcommand{\LaTeX}{\textrm{\Oldlatex}}
    % Document parameters
    % Document title
    \title{\vspace{-12ex}}
    \date{\vspace{-11ex}}
    
    
    
    
    
% Pygments definitions
\makeatletter
\def\PY@reset{\let\PY@it=\relax \let\PY@bf=\relax%
    \let\PY@ul=\relax \let\PY@tc=\relax%
    \let\PY@bc=\relax \let\PY@ff=\relax}
\def\PY@tok#1{\csname PY@tok@#1\endcsname}
\def\PY@toks#1+{\ifx\relax#1\empty\else%
    \PY@tok{#1}\expandafter\PY@toks\fi}
\def\PY@do#1{\PY@bc{\PY@tc{\PY@ul{%
    \PY@it{\PY@bf{\PY@ff{#1}}}}}}}
\def\PY#1#2{\PY@reset\PY@toks#1+\relax+\PY@do{#2}}

\expandafter\def\csname PY@tok@w\endcsname{\def\PY@tc##1{\textcolor[rgb]{0.73,0.73,0.73}{##1}}}
\expandafter\def\csname PY@tok@c\endcsname{\let\PY@it=\textit\def\PY@tc##1{\textcolor[rgb]{0.25,0.50,0.50}{##1}}}
\expandafter\def\csname PY@tok@cp\endcsname{\def\PY@tc##1{\textcolor[rgb]{0.74,0.48,0.00}{##1}}}
\expandafter\def\csname PY@tok@k\endcsname{\let\PY@bf=\textbf\def\PY@tc##1{\textcolor[rgb]{0.00,0.50,0.00}{##1}}}
\expandafter\def\csname PY@tok@kp\endcsname{\def\PY@tc##1{\textcolor[rgb]{0.00,0.50,0.00}{##1}}}
\expandafter\def\csname PY@tok@kt\endcsname{\def\PY@tc##1{\textcolor[rgb]{0.69,0.00,0.25}{##1}}}
\expandafter\def\csname PY@tok@o\endcsname{\def\PY@tc##1{\textcolor[rgb]{0.40,0.40,0.40}{##1}}}
\expandafter\def\csname PY@tok@ow\endcsname{\let\PY@bf=\textbf\def\PY@tc##1{\textcolor[rgb]{0.67,0.13,1.00}{##1}}}
\expandafter\def\csname PY@tok@nb\endcsname{\def\PY@tc##1{\textcolor[rgb]{0.00,0.50,0.00}{##1}}}
\expandafter\def\csname PY@tok@nf\endcsname{\def\PY@tc##1{\textcolor[rgb]{0.00,0.00,1.00}{##1}}}
\expandafter\def\csname PY@tok@nc\endcsname{\let\PY@bf=\textbf\def\PY@tc##1{\textcolor[rgb]{0.00,0.00,1.00}{##1}}}
\expandafter\def\csname PY@tok@nn\endcsname{\let\PY@bf=\textbf\def\PY@tc##1{\textcolor[rgb]{0.00,0.00,1.00}{##1}}}
\expandafter\def\csname PY@tok@ne\endcsname{\let\PY@bf=\textbf\def\PY@tc##1{\textcolor[rgb]{0.82,0.25,0.23}{##1}}}
\expandafter\def\csname PY@tok@nv\endcsname{\def\PY@tc##1{\textcolor[rgb]{0.10,0.09,0.49}{##1}}}
\expandafter\def\csname PY@tok@no\endcsname{\def\PY@tc##1{\textcolor[rgb]{0.53,0.00,0.00}{##1}}}
\expandafter\def\csname PY@tok@nl\endcsname{\def\PY@tc##1{\textcolor[rgb]{0.63,0.63,0.00}{##1}}}
\expandafter\def\csname PY@tok@ni\endcsname{\let\PY@bf=\textbf\def\PY@tc##1{\textcolor[rgb]{0.60,0.60,0.60}{##1}}}
\expandafter\def\csname PY@tok@na\endcsname{\def\PY@tc##1{\textcolor[rgb]{0.49,0.56,0.16}{##1}}}
\expandafter\def\csname PY@tok@nt\endcsname{\let\PY@bf=\textbf\def\PY@tc##1{\textcolor[rgb]{0.00,0.50,0.00}{##1}}}
\expandafter\def\csname PY@tok@nd\endcsname{\def\PY@tc##1{\textcolor[rgb]{0.67,0.13,1.00}{##1}}}
\expandafter\def\csname PY@tok@s\endcsname{\def\PY@tc##1{\textcolor[rgb]{0.73,0.13,0.13}{##1}}}
\expandafter\def\csname PY@tok@sd\endcsname{\let\PY@it=\textit\def\PY@tc##1{\textcolor[rgb]{0.73,0.13,0.13}{##1}}}
\expandafter\def\csname PY@tok@si\endcsname{\let\PY@bf=\textbf\def\PY@tc##1{\textcolor[rgb]{0.73,0.40,0.53}{##1}}}
\expandafter\def\csname PY@tok@se\endcsname{\let\PY@bf=\textbf\def\PY@tc##1{\textcolor[rgb]{0.73,0.40,0.13}{##1}}}
\expandafter\def\csname PY@tok@sr\endcsname{\def\PY@tc##1{\textcolor[rgb]{0.73,0.40,0.53}{##1}}}
\expandafter\def\csname PY@tok@ss\endcsname{\def\PY@tc##1{\textcolor[rgb]{0.10,0.09,0.49}{##1}}}
\expandafter\def\csname PY@tok@sx\endcsname{\def\PY@tc##1{\textcolor[rgb]{0.00,0.50,0.00}{##1}}}
\expandafter\def\csname PY@tok@m\endcsname{\def\PY@tc##1{\textcolor[rgb]{0.40,0.40,0.40}{##1}}}
\expandafter\def\csname PY@tok@gh\endcsname{\let\PY@bf=\textbf\def\PY@tc##1{\textcolor[rgb]{0.00,0.00,0.50}{##1}}}
\expandafter\def\csname PY@tok@gu\endcsname{\let\PY@bf=\textbf\def\PY@tc##1{\textcolor[rgb]{0.50,0.00,0.50}{##1}}}
\expandafter\def\csname PY@tok@gd\endcsname{\def\PY@tc##1{\textcolor[rgb]{0.63,0.00,0.00}{##1}}}
\expandafter\def\csname PY@tok@gi\endcsname{\def\PY@tc##1{\textcolor[rgb]{0.00,0.63,0.00}{##1}}}
\expandafter\def\csname PY@tok@gr\endcsname{\def\PY@tc##1{\textcolor[rgb]{1.00,0.00,0.00}{##1}}}
\expandafter\def\csname PY@tok@ge\endcsname{\let\PY@it=\textit}
\expandafter\def\csname PY@tok@gs\endcsname{\let\PY@bf=\textbf}
\expandafter\def\csname PY@tok@gp\endcsname{\let\PY@bf=\textbf\def\PY@tc##1{\textcolor[rgb]{0.00,0.00,0.50}{##1}}}
\expandafter\def\csname PY@tok@go\endcsname{\def\PY@tc##1{\textcolor[rgb]{0.53,0.53,0.53}{##1}}}
\expandafter\def\csname PY@tok@gt\endcsname{\def\PY@tc##1{\textcolor[rgb]{0.00,0.27,0.87}{##1}}}
\expandafter\def\csname PY@tok@err\endcsname{\def\PY@bc##1{\setlength{\fboxsep}{0pt}\fcolorbox[rgb]{1.00,0.00,0.00}{1,1,1}{\strut ##1}}}
\expandafter\def\csname PY@tok@kc\endcsname{\let\PY@bf=\textbf\def\PY@tc##1{\textcolor[rgb]{0.00,0.50,0.00}{##1}}}
\expandafter\def\csname PY@tok@kd\endcsname{\let\PY@bf=\textbf\def\PY@tc##1{\textcolor[rgb]{0.00,0.50,0.00}{##1}}}
\expandafter\def\csname PY@tok@kn\endcsname{\let\PY@bf=\textbf\def\PY@tc##1{\textcolor[rgb]{0.00,0.50,0.00}{##1}}}
\expandafter\def\csname PY@tok@kr\endcsname{\let\PY@bf=\textbf\def\PY@tc##1{\textcolor[rgb]{0.00,0.50,0.00}{##1}}}
\expandafter\def\csname PY@tok@bp\endcsname{\def\PY@tc##1{\textcolor[rgb]{0.00,0.50,0.00}{##1}}}
\expandafter\def\csname PY@tok@fm\endcsname{\def\PY@tc##1{\textcolor[rgb]{0.00,0.00,1.00}{##1}}}
\expandafter\def\csname PY@tok@vc\endcsname{\def\PY@tc##1{\textcolor[rgb]{0.10,0.09,0.49}{##1}}}
\expandafter\def\csname PY@tok@vg\endcsname{\def\PY@tc##1{\textcolor[rgb]{0.10,0.09,0.49}{##1}}}
\expandafter\def\csname PY@tok@vi\endcsname{\def\PY@tc##1{\textcolor[rgb]{0.10,0.09,0.49}{##1}}}
\expandafter\def\csname PY@tok@vm\endcsname{\def\PY@tc##1{\textcolor[rgb]{0.10,0.09,0.49}{##1}}}
\expandafter\def\csname PY@tok@sa\endcsname{\def\PY@tc##1{\textcolor[rgb]{0.73,0.13,0.13}{##1}}}
\expandafter\def\csname PY@tok@sb\endcsname{\def\PY@tc##1{\textcolor[rgb]{0.73,0.13,0.13}{##1}}}
\expandafter\def\csname PY@tok@sc\endcsname{\def\PY@tc##1{\textcolor[rgb]{0.73,0.13,0.13}{##1}}}
\expandafter\def\csname PY@tok@dl\endcsname{\def\PY@tc##1{\textcolor[rgb]{0.73,0.13,0.13}{##1}}}
\expandafter\def\csname PY@tok@s2\endcsname{\def\PY@tc##1{\textcolor[rgb]{0.73,0.13,0.13}{##1}}}
\expandafter\def\csname PY@tok@sh\endcsname{\def\PY@tc##1{\textcolor[rgb]{0.73,0.13,0.13}{##1}}}
\expandafter\def\csname PY@tok@s1\endcsname{\def\PY@tc##1{\textcolor[rgb]{0.73,0.13,0.13}{##1}}}
\expandafter\def\csname PY@tok@mb\endcsname{\def\PY@tc##1{\textcolor[rgb]{0.40,0.40,0.40}{##1}}}
\expandafter\def\csname PY@tok@mf\endcsname{\def\PY@tc##1{\textcolor[rgb]{0.40,0.40,0.40}{##1}}}
\expandafter\def\csname PY@tok@mh\endcsname{\def\PY@tc##1{\textcolor[rgb]{0.40,0.40,0.40}{##1}}}
\expandafter\def\csname PY@tok@mi\endcsname{\def\PY@tc##1{\textcolor[rgb]{0.40,0.40,0.40}{##1}}}
\expandafter\def\csname PY@tok@il\endcsname{\def\PY@tc##1{\textcolor[rgb]{0.40,0.40,0.40}{##1}}}
\expandafter\def\csname PY@tok@mo\endcsname{\def\PY@tc##1{\textcolor[rgb]{0.40,0.40,0.40}{##1}}}
\expandafter\def\csname PY@tok@ch\endcsname{\let\PY@it=\textit\def\PY@tc##1{\textcolor[rgb]{0.25,0.50,0.50}{##1}}}
\expandafter\def\csname PY@tok@cm\endcsname{\let\PY@it=\textit\def\PY@tc##1{\textcolor[rgb]{0.25,0.50,0.50}{##1}}}
\expandafter\def\csname PY@tok@cpf\endcsname{\let\PY@it=\textit\def\PY@tc##1{\textcolor[rgb]{0.25,0.50,0.50}{##1}}}
\expandafter\def\csname PY@tok@c1\endcsname{\let\PY@it=\textit\def\PY@tc##1{\textcolor[rgb]{0.25,0.50,0.50}{##1}}}
\expandafter\def\csname PY@tok@cs\endcsname{\let\PY@it=\textit\def\PY@tc##1{\textcolor[rgb]{0.25,0.50,0.50}{##1}}}

\def\PYZbs{\char`\\}
\def\PYZus{\char`\_}
\def\PYZob{\char`\{}
\def\PYZcb{\char`\}}
\def\PYZca{\char`\^}
\def\PYZam{\char`\&}
\def\PYZlt{\char`\<}
\def\PYZgt{\char`\>}
\def\PYZsh{\char`\#}
\def\PYZpc{\char`\%}
\def\PYZdl{\char`\$}
\def\PYZhy{\char`\-}
\def\PYZsq{\char`\'}
\def\PYZdq{\char`\"}
\def\PYZti{\char`\~}
% for compatibility with earlier versions
\def\PYZat{@}
\def\PYZlb{[}
\def\PYZrb{]}
\makeatother


    % For linebreaks inside Verbatim environment from package fancyvrb. 
    \makeatletter
        \newbox\Wrappedcontinuationbox 
        \newbox\Wrappedvisiblespacebox 
        \newcommand*\Wrappedvisiblespace {\textcolor{red}{\textvisiblespace}} 
        \newcommand*\Wrappedcontinuationsymbol {\textcolor{red}{\llap{\tiny$\m@th\hookrightarrow$}}} 
        \newcommand*\Wrappedcontinuationindent {3ex } 
        \newcommand*\Wrappedafterbreak {\kern\Wrappedcontinuationindent\copy\Wrappedcontinuationbox} 
        % Take advantage of the already applied Pygments mark-up to insert 
        % potential linebreaks for TeX processing. 
        %        {, <, #, %, $, ' and ": go to next line. 
        %        _, }, ^, &, >, - and ~: stay at end of broken line. 
        % Use of \textquotesingle for straight quote. 
        \newcommand*\Wrappedbreaksatspecials {% 
            \def\PYGZus{\discretionary{\char`\_}{\Wrappedafterbreak}{\char`\_}}% 
            \def\PYGZob{\discretionary{}{\Wrappedafterbreak\char`\{}{\char`\{}}% 
            \def\PYGZcb{\discretionary{\char`\}}{\Wrappedafterbreak}{\char`\}}}% 
            \def\PYGZca{\discretionary{\char`\^}{\Wrappedafterbreak}{\char`\^}}% 
            \def\PYGZam{\discretionary{\char`\&}{\Wrappedafterbreak}{\char`\&}}% 
            \def\PYGZlt{\discretionary{}{\Wrappedafterbreak\char`\<}{\char`\<}}% 
            \def\PYGZgt{\discretionary{\char`\>}{\Wrappedafterbreak}{\char`\>}}% 
            \def\PYGZsh{\discretionary{}{\Wrappedafterbreak\char`\#}{\char`\#}}% 
            \def\PYGZpc{\discretionary{}{\Wrappedafterbreak\char`\%}{\char`\%}}% 
            \def\PYGZdl{\discretionary{}{\Wrappedafterbreak\char`\$}{\char`\$}}% 
            \def\PYGZhy{\discretionary{\char`\-}{\Wrappedafterbreak}{\char`\-}}% 
            \def\PYGZsq{\discretionary{}{\Wrappedafterbreak\textquotesingle}{\textquotesingle}}% 
            \def\PYGZdq{\discretionary{}{\Wrappedafterbreak\char`\"}{\char`\"}}% 
            \def\PYGZti{\discretionary{\char`\~}{\Wrappedafterbreak}{\char`\~}}% 
        } 
        % Some characters . , ; ? ! / are not pygmentized. 
        % This macro makes them "active" and they will insert potential linebreaks 
        \newcommand*\Wrappedbreaksatpunct {% 
            \lccode`\~`\.\lowercase{\def~}{\discretionary{\hbox{\char`\.}}{\Wrappedafterbreak}{\hbox{\char`\.}}}% 
            \lccode`\~`\,\lowercase{\def~}{\discretionary{\hbox{\char`\,}}{\Wrappedafterbreak}{\hbox{\char`\,}}}% 
            \lccode`\~`\;\lowercase{\def~}{\discretionary{\hbox{\char`\;}}{\Wrappedafterbreak}{\hbox{\char`\;}}}% 
            \lccode`\~`\:\lowercase{\def~}{\discretionary{\hbox{\char`\:}}{\Wrappedafterbreak}{\hbox{\char`\:}}}% 
            \lccode`\~`\?\lowercase{\def~}{\discretionary{\hbox{\char`\?}}{\Wrappedafterbreak}{\hbox{\char`\?}}}% 
            \lccode`\~`\!\lowercase{\def~}{\discretionary{\hbox{\char`\!}}{\Wrappedafterbreak}{\hbox{\char`\!}}}% 
            \lccode`\~`\/\lowercase{\def~}{\discretionary{\hbox{\char`\/}}{\Wrappedafterbreak}{\hbox{\char`\/}}}% 
            \catcode`\.\active
            \catcode`\,\active 
            \catcode`\;\active
            \catcode`\:\active
            \catcode`\?\active
            \catcode`\!\active
            \catcode`\/\active 
            \lccode`\~`\~ 	
        }
    \makeatother

    \let\OriginalVerbatim=\Verbatim
    \makeatletter
    \renewcommand{\Verbatim}[1][1]{%
        %\parskip\z@skip
        \sbox\Wrappedcontinuationbox {\Wrappedcontinuationsymbol}%
        \sbox\Wrappedvisiblespacebox {\FV@SetupFont\Wrappedvisiblespace}%
        \def\FancyVerbFormatLine ##1{\hsize\linewidth
            \vtop{\raggedright\hyphenpenalty\z@\exhyphenpenalty\z@
                \doublehyphendemerits\z@\finalhyphendemerits\z@
                \strut ##1\strut}%
        }%
        % If the linebreak is at a space, the latter will be displayed as visible
        % space at end of first line, and a continuation symbol starts next line.
        % Stretch/shrink are however usually zero for typewriter font.
        \def\FV@Space {%
            \nobreak\hskip\z@ plus\fontdimen3\font minus\fontdimen4\font
            \discretionary{\copy\Wrappedvisiblespacebox}{\Wrappedafterbreak}
            {\kern\fontdimen2\font}%
        }%
        
        % Allow breaks at special characters using \PYG... macros.
        \Wrappedbreaksatspecials
        % Breaks at punctuation characters . , ; ? ! and / need catcode=\active 	
        \OriginalVerbatim[#1,codes*=\Wrappedbreaksatpunct]%
    }
    \makeatother

    % Exact colors from NB
    \definecolor{incolor}{HTML}{303F9F}
    \definecolor{outcolor}{HTML}{D84315}
    \definecolor{cellborder}{HTML}{CFCFCF}
    \definecolor{cellbackground}{HTML}{F7F7F7}
    
    % prompt
    \newcommand{\prompt}[4]{
        \llap{{\color{#2}[#3]: #4}}\vspace{-1.25em}
    }
    

    
    % Prevent overflowing lines due to hard-to-break entities
    \sloppy 
    % Setup hyperref package
    \hypersetup{
      breaklinks=true,  % so long urls are correctly broken across lines
      colorlinks=true,
      urlcolor=urlcolor,
      linkcolor=linkcolor,
      citecolor=citecolor,
      }
    % Slightly bigger margins than the latex defaults
    
    \geometry{verbose,tmargin=1in,bmargin=1in,lmargin=1in,rmargin=1in}
    
    

    \begin{document}
    
    
    \maketitle
    
    

    
    \hypertarget{problem-3-code-results-parts-b---d}{%
\subsubsection*{Problem 3 Code \& Results (Parts b -
d)}\label{problem-3-code-results-parts-b---d}}

%\underline{Note}: My code/solution (below) is also provided in
%``\(\texttt{meanvariance.ipynb}\)'' notebook. However, I put it in this
%separate notebook so that my code and results are separate from the
%professor's code. The ``\(\texttt{meanvariance.ipynb}\)'' notebook
%contains a lot of the professor's work in the beginning, it could cause
%confusion.\\

    \begin{tcolorbox}[breakable, size=fbox, boxrule=1pt, pad at break*=1mm,colback=cellbackground, colframe=cellborder]
\prompt{In}{incolor}{1}{\hspace{4pt}}
\begin{Verbatim}[commandchars=\\\{\}]
\PY{k+kn}{import} \PY{n+nn}{numpy} \PY{k}{as} \PY{n+nn}{np}
\PY{k+kn}{import} \PY{n+nn}{scipy}\PY{n+nn}{.}\PY{n+nn}{io} \PY{k}{as} \PY{n+nn}{sio}
\PY{k+kn}{import} \PY{n+nn}{cvxpy} \PY{k}{as} \PY{n+nn}{cvx}
\PY{k+kn}{import} \PY{n+nn}{matplotlib}\PY{n+nn}{.}\PY{n+nn}{pyplot} \PY{k}{as} \PY{n+nn}{plt}
\PY{k+kn}{import} \PY{n+nn}{pandas} \PY{k}{as} \PY{n+nn}{pd}
\PY{o}{\PYZpc{}}\PY{k}{matplotlib} inline
\end{Verbatim}
\end{tcolorbox}

    \begin{tcolorbox}[breakable, size=fbox, boxrule=1pt, pad at break*=1mm,colback=cellbackground, colframe=cellborder]
\prompt{In}{incolor}{2}{\hspace{4pt}}
\begin{Verbatim}[commandchars=\\\{\}]
\PY{n}{mat\PYZus{}contents} \PY{o}{=} \PY{n}{sio}\PY{o}{.}\PY{n}{loadmat}\PY{p}{(}\PY{l+s+s1}{\PYZsq{}}\PY{l+s+s1}{meanvariance.mat}\PY{l+s+s1}{\PYZsq{}}\PY{p}{)}
\PY{n}{mu} \PY{o}{=} \PY{n}{mat\PYZus{}contents}\PY{p}{[}\PY{l+s+s1}{\PYZsq{}}\PY{l+s+s1}{mu}\PY{l+s+s1}{\PYZsq{}}\PY{p}{]}
\PY{n}{S} \PY{o}{=} \PY{n}{np}\PY{o}{.}\PY{n}{matrix}\PY{p}{(}\PY{n}{mat\PYZus{}contents}\PY{p}{[}\PY{l+s+s1}{\PYZsq{}}\PY{l+s+s1}{S}\PY{l+s+s1}{\PYZsq{}}\PY{p}{]}\PY{p}{)}
\PY{n}{d} \PY{o}{=} \PY{n}{np}\PY{o}{.}\PY{n}{size}\PY{p}{(}\PY{n}{mu}\PY{p}{)}
\end{Verbatim}
\end{tcolorbox}
\newline
\newline
\textbf{\underline{Part b}:} \\

    \begin{tcolorbox}[breakable, size=fbox, boxrule=1pt, pad at break*=1mm,colback=cellbackground, colframe=cellborder]
\prompt{In}{incolor}{3}{\hspace{4pt}}
\begin{Verbatim}[commandchars=\\\{\}]
\PY{n}{J} \PY{o}{=} \PY{n}{np}\PY{o}{.}\PY{n}{ones}\PY{p}{(}\PY{p}{(}\PY{n+nb}{len}\PY{p}{(}\PY{n}{S}\PY{p}{)}\PY{p}{,}\PY{l+m+mi}{1}\PY{p}{)}\PY{p}{)}    \PY{c+c1}{\PYZsh{} d x 1 ones vector}
\PY{n}{S\PYZus{}inv} \PY{o}{=} \PY{n}{np}\PY{o}{.}\PY{n}{linalg}\PY{o}{.}\PY{n}{inv}\PY{p}{(}\PY{n}{S}\PY{p}{)}   \PY{c+c1}{\PYZsh{} inverse of covariance matrix}
\end{Verbatim}
\end{tcolorbox}

    \begin{tcolorbox}[breakable, size=fbox, boxrule=1pt, pad at break*=1mm,colback=cellbackground, colframe=cellborder]
\prompt{In}{incolor}{4}{\hspace{4pt}}
\begin{Verbatim}[commandchars=\\\{\}]
\PY{n}{x\PYZus{}numerator} \PY{o}{=} \PY{n}{np}\PY{o}{.}\PY{n}{matmul}\PY{p}{(}\PY{n}{S\PYZus{}inv}\PY{p}{,} \PY{n}{J}\PY{p}{)}     \PY{c+c1}{\PYZsh{} numerator of the optimal solution}
\PY{n}{JS\PYZus{}inv} \PY{o}{=} \PY{n}{J}\PY{o}{.}\PY{n}{T}\PY{o}{*}\PY{n}{S\PYZus{}inv}
\PY{n}{x\PYZus{}denominator} \PY{o}{=} \PY{n}{np}\PY{o}{.}\PY{n}{matmul}\PY{p}{(}\PY{n}{JS\PYZus{}inv}\PY{p}{,} \PY{n}{J}\PY{p}{)}  \PY{c+c1}{\PYZsh{} denominator of the optimal solution}

\PY{n}{x\PYZus{}min} \PY{o}{=} \PY{n}{x\PYZus{}numerator}\PY{o}{/}\PY{n}{x\PYZus{}denominator}     \PY{c+c1}{\PYZsh{} optimal solution (min variance portfolio)}
\PY{n}{r\PYZus{}min} \PY{o}{=} \PY{n}{np}\PY{o}{.}\PY{n}{dot}\PY{p}{(}\PY{n}{mu}\PY{o}{.}\PY{n}{T}\PY{p}{,} \PY{n}{x\PYZus{}min}\PY{p}{)}\PY{o}{.}\PY{n}{item}\PY{p}{(}\PY{l+m+mi}{0}\PY{p}{)}   \PY{c+c1}{\PYZsh{} minimum expected return}
\end{Verbatim}
\end{tcolorbox}

    \begin{tcolorbox}[breakable, size=fbox, boxrule=1pt, pad at break*=1mm,colback=cellbackground, colframe=cellborder]
\prompt{In}{incolor}{5}{\hspace{4pt}}
\begin{Verbatim}[commandchars=\\\{\}]
\PY{n}{opt\PYZus{}val} \PY{o}{=} \PY{n}{x\PYZus{}min}\PY{o}{.}\PY{n}{T}\PY{o}{*}\PY{n}{S}\PY{o}{*}\PY{n}{x\PYZus{}min}           \PY{c+c1}{\PYZsh{} minimum variance (i.e. optimal objective value)}

\PY{n}{sigma2\PYZus{}min} \PY{o}{=} \PY{n}{opt\PYZus{}val}\PY{o}{.}\PY{n}{item}\PY{p}{(}\PY{l+m+mi}{0}\PY{p}{)}        \PY{c+c1}{\PYZsh{} extracts the scalar value from the 1x1 matrix}
\end{Verbatim}
\end{tcolorbox}

    \begin{tcolorbox}[breakable, size=fbox, boxrule=1pt, pad at break*=1mm,colback=cellbackground, colframe=cellborder]
\prompt{In}{incolor}{6}{\hspace{4pt}}
\begin{Verbatim}[commandchars=\\\{\}]
\PY{n+nb}{print}\PY{p}{(}\PY{l+s+s1}{\PYZsq{}}\PY{l+s+s1}{Minimum Variance (opt value):  }\PY{l+s+s1}{\PYZsq{}}\PY{p}{,} \PY{n}{sigma2\PYZus{}min}\PY{p}{)}
\PY{n+nb}{print}\PY{p}{(}\PY{l+s+s1}{\PYZsq{}}\PY{l+s+se}{\PYZbs{}n}\PY{l+s+s1}{Minimum Return (r\PYZus{}min):  }\PY{l+s+s1}{\PYZsq{}}\PY{p}{,} \PY{n}{r\PYZus{}min}\PY{p}{)}
\PY{c+c1}{\PYZsh{}print(\PYZsq{}\PYZbs{}nMinimum Variance Portfolio x (optimal solution):\PYZbs{}n\PYZbs{}n\PYZsq{}, x\PYZus{}min)}
\end{Verbatim}
\end{tcolorbox}

    \begin{Verbatim}[commandchars=\\\{\}]
Minimum Variance (opt value):   5.850607343853076e-11

Minimum Return (r\_min):   0.05152761072308839
\end{Verbatim}
\vspace{-0.15in}
\text{}
\newline
\textbf{\underline{Part c}:} 
\newline
Want to solve the following system of equations for \(x\) and \(u\)
(critical values)

\[
\underbrace{\begin{bmatrix}
\begin{bmatrix} \mu^\top \\ \mathbf{1}^\top \end{bmatrix}  & \mathbf{0}\\
2S & -\begin{bmatrix} \mu & \mathbf{1} \end{bmatrix} 
\end{bmatrix}}_{M}
\underbrace{\begin{bmatrix}
x \\ u
\end{bmatrix}}_{z}
= \underbrace{\begin{bmatrix}
\begin{bmatrix} 1.1r_{min} \\ 1 \end{bmatrix} \\
0
\end{bmatrix}}_{q}
\]

    \begin{tcolorbox}[breakable, size=fbox, boxrule=1pt, pad at break*=1mm,colback=cellbackground, colframe=cellborder]
\prompt{In}{incolor}{7}{\hspace{4pt}}
\begin{Verbatim}[commandchars=\\\{\}]
\PY{n}{M11} \PY{o}{=} \PY{n}{np}\PY{o}{.}\PY{n}{concatenate}\PY{p}{(}\PY{p}{(}\PY{n}{mu}\PY{o}{.}\PY{n}{T}\PY{p}{,} \PY{n}{J}\PY{o}{.}\PY{n}{T}\PY{p}{)}\PY{p}{,}\PY{n}{axis}\PY{o}{=}\PY{l+m+mi}{0}\PY{p}{)}
\PY{n}{M12} \PY{o}{=} \PY{n}{np}\PY{o}{.}\PY{n}{zeros}\PY{p}{(}\PY{p}{(}\PY{n+nb}{len}\PY{p}{(}\PY{n}{M11}\PY{p}{)}\PY{p}{,}\PY{n+nb}{len}\PY{p}{(}\PY{n}{M11}\PY{p}{)}\PY{p}{)}\PY{p}{)}
\PY{n}{M21} \PY{o}{=} \PY{l+m+mi}{2}\PY{o}{*}\PY{n}{S}
\PY{n}{M22} \PY{o}{=} \PY{n}{np}\PY{o}{.}\PY{n}{concatenate}\PY{p}{(}\PY{p}{(}\PY{n}{mu}\PY{p}{,} \PY{n}{J}\PY{p}{)}\PY{p}{,}\PY{n}{axis}\PY{o}{=}\PY{l+m+mi}{1}\PY{p}{)}

\PY{n}{M1} \PY{o}{=} \PY{n}{np}\PY{o}{.}\PY{n}{concatenate}\PY{p}{(}\PY{p}{(}\PY{n}{M11}\PY{p}{,} \PY{n}{M12}\PY{p}{)}\PY{p}{,} \PY{n}{axis}\PY{o}{=}\PY{l+m+mi}{1}\PY{p}{)}
\PY{n}{M2} \PY{o}{=} \PY{n}{np}\PY{o}{.}\PY{n}{concatenate}\PY{p}{(}\PY{p}{(}\PY{n}{M21}\PY{p}{,} \PY{n}{M22}\PY{p}{)}\PY{p}{,}\PY{n}{axis}\PY{o}{=}\PY{l+m+mi}{1}\PY{p}{)}
\PY{n}{M} \PY{o}{=} \PY{n}{np}\PY{o}{.}\PY{n}{concatenate}\PY{p}{(}\PY{p}{(}\PY{n}{M1}\PY{p}{,}\PY{n}{M2}\PY{p}{)}\PY{p}{,}\PY{n}{axis}\PY{o}{=}\PY{l+m+mi}{0}\PY{p}{)}

\PY{n}{b} \PY{o}{=} \PY{n}{np}\PY{o}{.}\PY{n}{array}\PY{p}{(}\PY{p}{[}\PY{p}{[}\PY{l+m+mf}{1.1}\PY{o}{*}\PY{n}{r\PYZus{}min}\PY{p}{]}\PY{p}{,} \PY{p}{[}\PY{l+m+mi}{1}\PY{p}{]}\PY{p}{]}\PY{p}{)}
\PY{n}{q} \PY{o}{=} \PY{n}{np}\PY{o}{.}\PY{n}{concatenate}\PY{p}{(}\PY{p}{(}\PY{n}{b}\PY{p}{,} \PY{n}{np}\PY{o}{.}\PY{n}{zeros}\PY{p}{(}\PY{p}{(}\PY{n+nb}{len}\PY{p}{(}\PY{n}{S}\PY{p}{)}\PY{p}{,}\PY{l+m+mi}{1}\PY{p}{)}\PY{p}{)}\PY{p}{)}\PY{p}{,} \PY{n}{axis}\PY{o}{=}\PY{l+m+mi}{0}\PY{p}{)}
\end{Verbatim}
\end{tcolorbox}

    \begin{tcolorbox}[breakable, size=fbox, boxrule=1pt, pad at break*=1mm,colback=cellbackground, colframe=cellborder]
\prompt{In}{incolor}{8}{\hspace{4pt}}
\begin{Verbatim}[commandchars=\\\{\}]
\PY{n}{z} \PY{o}{=} \PY{n}{np}\PY{o}{.}\PY{n}{linalg}\PY{o}{.}\PY{n}{solve}\PY{p}{(}\PY{n}{M}\PY{p}{,}\PY{n}{q}\PY{p}{)}
\PY{n}{x\PYZus{}opt} \PY{o}{=} \PY{n}{z}\PY{p}{[}\PY{l+m+mi}{0}\PY{p}{:}\PY{n}{d}\PY{p}{]}
\end{Verbatim}
\end{tcolorbox}
\text{}
\newline
Alternative approach to solving for \(x^*\) (Determined using the
lecture notes on the optimal solution to equality constrained QPs): \\

\(x^* = S^{-1}\begin{bmatrix} \mu & \mathbf{1} \end{bmatrix} \left(\begin{bmatrix} \mu^\top \\ \mathbf{1}^\top \end{bmatrix} S^{-1} \begin{bmatrix} \mu & \mathbf{1} \end{bmatrix}\right)^{-1} \begin{bmatrix} 1.1r_{min} \\ 1 \end{bmatrix}\)

    \begin{tcolorbox}[breakable, size=fbox, boxrule=1pt, pad at break*=1mm,colback=cellbackground, colframe=cellborder]
\prompt{In}{incolor}{9}{\hspace{4pt}}
\begin{Verbatim}[commandchars=\\\{\}]
\PY{c+c1}{\PYZsh{} Computes the expression above for optimal x (primal critical values using gradient condition)}

\PY{n}{x\PYZus{}left} \PY{o}{=} \PY{n}{S\PYZus{}inv}\PY{o}{*}\PY{n}{M22}
\PY{n}{x\PYZus{}mid} \PY{o}{=} \PY{n}{np}\PY{o}{.}\PY{n}{linalg}\PY{o}{.}\PY{n}{inv}\PY{p}{(}\PY{n}{M11}\PY{o}{*}\PY{n}{S\PYZus{}inv}\PY{o}{*}\PY{n}{M22}\PY{p}{)}
\PY{n}{x\PYZus{}right} \PY{o}{=} \PY{n}{b}

\PY{n}{x\PYZus{}optimal} \PY{o}{=} \PY{n}{x\PYZus{}left}\PY{o}{*}\PY{n}{x\PYZus{}mid}\PY{o}{*}\PY{n}{x\PYZus{}right}
\end{Verbatim}
\end{tcolorbox}

    \begin{tcolorbox}[breakable, size=fbox, boxrule=1pt, pad at break*=1mm,colback=cellbackground, colframe=cellborder]
\prompt{In}{incolor}{10}{\hspace{4pt}}
\begin{Verbatim}[commandchars=\\\{\}]
\PY{n}{alpha1} \PY{o}{=} \PY{n}{r\PYZus{}min}
\PY{n}{alpha2} \PY{o}{=} \PY{n}{np}\PY{o}{.}\PY{n}{dot}\PY{p}{(}\PY{n}{mu}\PY{o}{.}\PY{n}{T}\PY{p}{,} \PY{n}{x\PYZus{}opt}\PY{p}{)}\PY{o}{.}\PY{n}{item}\PY{p}{(}\PY{l+m+mi}{0}\PY{p}{)}

\PY{n}{sigma2} \PY{o}{=} \PY{n}{np}\PY{o}{.}\PY{n}{zeros}\PY{p}{(}\PY{n+nb}{len}\PY{p}{(}\PY{n}{S}\PY{p}{)}\PY{p}{)} 
\PY{n}{r} \PY{o}{=} \PY{n}{np}\PY{o}{.}\PY{n}{zeros}\PY{p}{(}\PY{n+nb}{len}\PY{p}{(}\PY{n}{S}\PY{p}{)}\PY{p}{)}
\PY{n}{alpha} \PY{o}{=} \PY{n}{r\PYZus{}min}

\PY{k}{for} \PY{n}{i} \PY{o+ow}{in} \PY{n+nb}{range}\PY{p}{(}\PY{n+nb}{len}\PY{p}{(}\PY{n}{S}\PY{p}{)}\PY{p}{)}\PY{p}{:}
    \PY{n}{weight} \PY{o}{=} \PY{p}{(}\PY{n}{alpha2}\PY{o}{\PYZhy{}}\PY{n}{alpha}\PY{p}{)}\PY{o}{/}\PY{p}{(}\PY{n}{alpha2}\PY{o}{\PYZhy{}}\PY{n}{alpha1}\PY{p}{)}
    \PY{n}{phi} \PY{o}{=} \PY{n}{weight}\PY{o}{*}\PY{n}{x\PYZus{}min} \PY{o}{+} \PY{p}{(}\PY{l+m+mi}{1}\PY{o}{\PYZhy{}}\PY{n}{weight}\PY{p}{)}\PY{o}{*}\PY{n}{x\PYZus{}opt}
    \PY{c+c1}{\PYZsh{}eff = np.concatenate((eff, phi), axis=1)}
    \PY{n}{sigma2}\PY{p}{[}\PY{n}{i}\PY{p}{]} \PY{o}{=} \PY{n}{phi}\PY{o}{.}\PY{n}{T}\PY{o}{*}\PY{n}{S}\PY{o}{*}\PY{n}{phi}
    \PY{n}{r}\PY{p}{[}\PY{n}{i}\PY{p}{]} \PY{o}{=} \PY{n}{alpha}
    \PY{n}{alpha} \PY{o}{+}\PY{o}{=} \PY{l+m+mf}{0.0001}
\end{Verbatim}
\end{tcolorbox}

    \begin{tcolorbox}[breakable, size=fbox, boxrule=1pt, pad at break*=1mm,colback=cellbackground, colframe=cellborder]
\prompt{In}{incolor}{11}{\hspace{4pt}}
\begin{Verbatim}[commandchars=\\\{\}]
\PY{n}{plt}\PY{o}{.}\PY{n}{title}\PY{p}{(}\PY{l+s+s1}{\PYZsq{}}\PY{l+s+s1}{Efficient Frontier}\PY{l+s+s1}{\PYZsq{}}\PY{p}{)}
\PY{n}{plt}\PY{o}{.}\PY{n}{xlabel}\PY{p}{(}\PY{l+s+sa}{r}\PY{l+s+s1}{\PYZsq{}}\PY{l+s+s1}{\PYZdl{}}\PY{l+s+s1}{\PYZbs{}}\PY{l+s+s1}{sigma\PYZca{}2\PYZdl{}}\PY{l+s+s1}{\PYZsq{}}\PY{p}{)}
\PY{n}{plt}\PY{o}{.}\PY{n}{ylabel}\PY{p}{(}\PY{l+s+sa}{r}\PY{l+s+s1}{\PYZsq{}}\PY{l+s+s1}{\PYZdl{}}\PY{l+s+s1}{\PYZbs{}}\PY{l+s+s1}{alpha\PYZdl{}}\PY{l+s+s1}{\PYZsq{}}\PY{p}{)}
\PY{n}{plt}\PY{o}{.}\PY{n}{plot}\PY{p}{(}\PY{n}{sigma2}\PY{p}{,} \PY{n}{r}\PY{p}{)}
\PY{n}{plt}\PY{o}{.}\PY{n}{show}\PY{p}{(}\PY{p}{)}
\end{Verbatim}
\end{tcolorbox}

    \begin{center}
    \adjustimage{max size={0.8\linewidth}{0.8\paperheight}}{output_15_0.png}
    \end{center}
    { \hspace*{\fill} \\}
    \vspace{-0.26in}
    \newline
    \textbf{\underline{Part d - Sharpe optimal portfolio}:}

    \begin{tcolorbox}[breakable, size=fbox, boxrule=1pt, pad at break*=1mm,colback=cellbackground, colframe=cellborder]
\prompt{In}{incolor}{12}{\hspace{4pt}}
\begin{Verbatim}[commandchars=\\\{\}]
\PY{n}{r\PYZus{}f} \PY{o}{=} \PY{l+m+mf}{0.04}               \PY{c+c1}{\PYZsh{}risk\PYZhy{}free rate}

\PY{n}{J} \PY{o}{=} \PY{n}{np}\PY{o}{.}\PY{n}{ones}\PY{p}{(}\PY{p}{(}\PY{n+nb}{len}\PY{p}{(}\PY{n}{mu}\PY{p}{)}\PY{p}{,}\PY{l+m+mi}{1}\PY{p}{)}\PY{p}{)} \PY{c+c1}{\PYZsh{} d x 1 ones vector }
\PY{n}{mu\PYZus{}ex} \PY{o}{=} \PY{n}{mu} \PY{o}{\PYZhy{}} \PY{n}{r\PYZus{}f}\PY{o}{*}\PY{n}{J}       \PY{c+c1}{\PYZsh{} vector of expected excess returns}

\PY{n}{x\PYZus{}num} \PY{o}{=} \PY{n}{S\PYZus{}inv}\PY{o}{*}\PY{n}{mu\PYZus{}ex}     
\PY{n}{x\PYZus{}den} \PY{o}{=} \PY{n}{mu\PYZus{}ex}\PY{o}{.}\PY{n}{T}\PY{o}{*}\PY{n}{S\PYZus{}inv}\PY{o}{*}\PY{n}{mu\PYZus{}ex}

\PY{n}{x\PYZus{}star} \PY{o}{=} \PY{n}{x\PYZus{}num}\PY{o}{/}\PY{n}{x\PYZus{}den}     \PY{c+c1}{\PYZsh{} Optimal solution }

\PY{n}{x\PYZus{}sh} \PY{o}{=} \PY{n}{x\PYZus{}star}\PY{o}{/}\PY{n}{np}\PY{o}{.}\PY{n}{dot}\PY{p}{(}\PY{n}{J}\PY{o}{.}\PY{n}{T}\PY{p}{,} \PY{n}{x\PYZus{}star}\PY{p}{)}\PY{o}{.}\PY{n}{item}\PY{p}{(}\PY{l+m+mi}{0}\PY{p}{)}    \PY{c+c1}{\PYZsh{} Sharpe optimal portfolio}
\end{Verbatim}
\end{tcolorbox}



    \begin{tcolorbox}[breakable, size=fbox, boxrule=1pt, pad at break*=1mm,colback=cellbackground, colframe=cellborder]
\prompt{In}{incolor}{13}{\hspace{4pt}}
\begin{Verbatim}[commandchars=\\\{\}]
\PY{n}{opt\PYZus{}variance} \PY{o}{=} \PY{n}{x\PYZus{}sh}\PY{o}{.}\PY{n}{T}\PY{o}{*}\PY{n}{S}\PY{o}{*}\PY{n}{x\PYZus{}sh}      \PY{c+c1}{\PYZsh{} variance on optimal Sharpe portfolio}
\PY{n}{ex\PYZus{}return} \PY{o}{=} \PY{n}{mu\PYZus{}ex}\PY{o}{.}\PY{n}{T}\PY{o}{*}\PY{n}{x\PYZus{}sh}          \PY{c+c1}{\PYZsh{} excess return on optimal Sharpe portfolio}
\PY{n}{sharpe\PYZus{}ratio} \PY{o}{=} \PY{n}{ex\PYZus{}return}\PY{o}{/}\PY{n}{np}\PY{o}{.}\PY{n}{sqrt}\PY{p}{(}\PY{n}{opt\PYZus{}variance}\PY{p}{)}

\PY{n+nb}{print}\PY{p}{(}\PY{l+s+s1}{\PYZsq{}}\PY{l+s+s1}{Optimal Sharpe portfolio statistics:}\PY{l+s+s1}{\PYZsq{}}\PY{p}{)}
\PY{n+nb}{print}\PY{p}{(}\PY{l+s+s1}{\PYZsq{}}\PY{l+s+se}{\PYZbs{}n}\PY{l+s+s1}{Variance:       }\PY{l+s+s1}{\PYZsq{}}\PY{p}{,} \PY{n}{opt\PYZus{}variance}\PY{o}{.}\PY{n}{item}\PY{p}{(}\PY{l+m+mi}{0}\PY{p}{)}\PY{p}{)}  
\PY{n+nb}{print}\PY{p}{(}\PY{l+s+s1}{\PYZsq{}}\PY{l+s+se}{\PYZbs{}n}\PY{l+s+s1}{Excess Return:  }\PY{l+s+s1}{\PYZsq{}}\PY{p}{,} \PY{n}{ex\PYZus{}return}\PY{o}{.}\PY{n}{item}\PY{p}{(}\PY{l+m+mi}{0}\PY{p}{)}\PY{p}{)} 
\PY{n+nb}{print}\PY{p}{(}\PY{l+s+s1}{\PYZsq{}}\PY{l+s+se}{\PYZbs{}n}\PY{l+s+s1}{Sharpe Ratio:   }\PY{l+s+s1}{\PYZsq{}}\PY{p}{,} \PY{n}{sharpe\PYZus{}ratio}\PY{o}{.}\PY{n}{item}\PY{p}{(}\PY{l+m+mi}{0}\PY{p}{)}\PY{p}{)}
\end{Verbatim}
\end{tcolorbox}

\begin{Verbatim}[commandchars=\\\{\}]
Optimal Sharpe portfolio statistics:

Variance:        1.3485956836610517e-10
Excess Return:   0.026571747427911182
Sharpe Ratio:    2288.1222961738217
\end{Verbatim}


    % Add a bibliography block to the postdoc
    
    
    
    \end{document}
